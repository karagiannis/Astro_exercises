%\documentclass[twocolumn]{article}
\documentclass[./exercises.tex]{subfiles}
\begin{document}

\textit{\textbf{Stone Edge observatoriet  } }
\begin{enumerate}
\item Skriv en beskrivning vad ni gjorde/vad som gjordes på observationerna
 och vilka resultat ni lyckades få ihop på max en sida.
Skriv också om problem och svårigheter med riktiga astronomiska observationer. \\

Stone Edge obsevatoriet finns norr om San Fransisco och ägs av en privatperson som äger
en vingård i anslutning, dock drivs det av Chicago universitetet.
Genomgång av Anna Arnadottir om hur hur man styr teleskopet såsom registrerad användare.

Telskopet styrs via kommunikationsprogrammet Slack.
Teleskopet låses före egen användning genom kommandot \verb+\lock+. Telskopdomen öppnas genom kommandot
\verb+\crack+. Tillgängliga kommandon listas med \verb+\help+.
Astronomiska objekt kan sökas i teleskopets databas t.ex. avseende M31 (Andromeda galaxen) genom kommandot
\verb+\find M31+. Objektet listas med ett nummer t.ex 1. Kommandot för att
peka teleskopet mot M31 blir då \verb+\point 1+. Man kan även peka teleskopet med
\verb+\point <rektascension> <deklination>+
Telskopet kan även centrera objektet i bilden med kommandot \verb+\pinpoint+.
Två preview bilder tas, den ena före korrigeringen den andra efter koorigeringen,
 som inte kommer att sparas.
 
Bilder tas med kommandot
\verb+\image <exponeringstid i selkunder>+\\
\verb+<pixelreduktion> <filter typ>+\\
\verb+<antal bilder>+. Pixelreduktionen anges med ett s.k. bind-tal. Om bind är 1 är varje pixel en data punkt och bilden blir
4K stor, om bind är 2 så kombineras två pixlar såsom en datapunkt. Försiktighet iakttages vid
inställningen av exponeringstiden då för lång exponeringstid
vid fotografering av ljusstarkt objekt kan skada CMOS-kameran.

Avseende filtertyper så finns olika konventioner för benämning av
våglängdsområdet varav en filtertyp kallas Sloan-filter som även kallas
SDSS-filter. Tillgängliga bandpass-filter är ``O3'' för ozonets emitterade våglängder,``g-band''
för våglängdsområdet ca $400$nm- till $530$nm,``r-band'' från ca $570$nm till $680$nm, ``'i-band'' från
ca $700$ nm till $830$nm (rött och infrarött), ``sii'' där våglängsområdet ej uppfattades, ``'clear'' vilket
betyder att inget filter kopplas in och slutligen ``'h-alpha''. H-alpha filtret är ett smalt bandpassfilter
på 654 nm och är en av övergångarna i Väte-atomens Balmer-serie, vilket tydligen skall vara fördelaktigt om man tittar på nebulosor.

Signalbehandling\\
Kamerans egenbrus (värmebrus) kan subtraheras bort om man tar en s.k. dark-bild. En bild tas då utan att öppna kamerans iris.
För att ta reda på variationer i hur enskilda pixlar förstärker ljuset för att kunna kompensera för detta tar man s.k.
flat-bilder mot en ljusblå himmel. Detta skall göras för varje bild.

För att kontrollera hur mycket brus pixelavläsningselektroniken tillför så att man kan kompensera för detta
så tas s.k. ``bias-bilder''.

Man kan jämföra den egna bilden med databasens bild genom kommandot \verb+\where+.
Det är bättre att ta många bilder med en sammanlagd totaltid än att ta hela bilden direkt då man 
fjärrstyr ett teleskop därför att man kan förlora internetuppkopplingen.

Fotografierna filer är av typen .fits och kan endast ses med speciella programvaror exemplevis
``Subaru''.
Vid observationen den 29/9-2022 så observerades bl.a. Andromeda, Jupiter, Hjärtnebulosan, Hästhuvudnebulosan.





\end{enumerate}


%\underbrace{}

% \hspace{1em}

%\begin{enumerate}[label=(\alph*)]
%\end{enumerate}

%$$
%  A = 
%  \begin{bmatrix}
%    1 & 0  & 2i\\
%    2i & 0 &  -4\\
%    -i &  0 & -2i\\
%  \end{bmatrix}
%$$

%\begin{flalign*}
%  A = 
%  \begin{bmatrix}
%    1 & 0  & 2i\\
%    2i & 0 &  -4\\
%    -i &  0 & -2i\\
%  \end{bmatrix}
%\end{flalign*}


%\begin{flalign*}
%\psi(x) = \begin{cases} Ae^{ikx}+Be^{-ikx} &\ \  x<-a \\
%                        Ce^{\kappa x}+De^{-\kappa x} &\ \ -a < x < a\\
%						Fe^{ikx} & \ \ x>a
%       \end{cases}
%\end{flalign*}
%[width=80mm,scale=0.7]
%\begin{figure}[H]
%  \includegraphics[width=\linewidth]{odd_finite.eps}
%  \caption{$z_0=0.1\pi,0.5\pi, 3\pi,7\pi$}
%  \label{fig4}
%\end{figure}
\end{document}












                                     
                                     



