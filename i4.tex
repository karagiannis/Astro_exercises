%\documentclass[twocolumn]{article}
\documentclass[./exercises.tex]{subfiles}
\begin{document}

\textit{\textbf{Inlämning 3  } }
\begin{enumerate}

\item \textit{17.41. What are the most prominent absorptions lines you would expect to find in the 
spectrum of a star with a surface temperature of}\\ 
\textit{(a) 35,000 K}\\
\textit{(b) 2800 K }\\
\textit{(c) 5800 K }\\

\textit{Briefly describe why these stars have such different spectra even though they have 
essentially the same chemical composition.  
(Tips: Figure 17-11 samt Table 17-2 i boken är användbara)}

\item\textit{18.8 Why is the daytime sky blue? Why are distant mountains purple? Why is the Sun red at 
sunset and sunrise? And how does that relate to the bluish color of reflection nebulae and 
the process of interstellar reddening?}\\

Himlen ser blå ut p.g.a. att att det kortvågigare blå ljuset sprids mot luftens molekyler och träffar ögat.
Spridningsfenomenet kallas Rayleigh spridning.
Himlen har en röd färgton p.g.a. att ljuset har färdats längre sträcka vid soluppgång och solnedgång det röda långvågigare ljuset har undgått spriding och nått ögat.\\
Den lila färgen kommer av kombinationen av blått ljus som spridits av luftensmolekyler samt rött ljus som färdats
utan att spridas. 


\item\textit{18.16 What is an evolutionary track? How can evolutionary tracks help us interpret the H-R 
diagram?}\\

\item\textit{19.4 Why do high-mass main-sequence stars have shorter lifetimes than those of lower 
mass?}\\
 
\item\textit{19.10 Explain why Earth is expected to become inhospitable to life long before the Sun 
becomes a red giant.}\\

\item\textit{20.24 Sketch the evolutionary track of the Sun on the H-R diagram from when it leaves the 
main-sequence to when it becomes a white dwarf.  Approximately how much mass will it 
have as a white dwarf and where did the lost mass go}\\
 

\end{enumerate}


%\underbrace{}

% \hspace{1em}

%\begin{enumerate}[label=(\alph*)]
%\end{enumerate}

%$$
%  A = 
%  \begin{bmatrix}
%    1 & 0  & 2i\\
%    2i & 0 &  -4\\
%    -i &  0 & -2i\\
%  \end{bmatrix}
%$$

%\begin{flalign*}
%  A = 
%  \begin{bmatrix}
%    1 & 0  & 2i\\
%    2i & 0 &  -4\\
%    -i &  0 & -2i\\
%  \end{bmatrix}
%\end{flalign*}


%\begin{flalign*}
%\psi(x) = \begin{cases} Ae^{ikx}+Be^{-ikx} &\ \  x<-a \\
%                        Ce^{\kappa x}+De^{-\kappa x} &\ \ -a < x < a\\
%						Fe^{ikx} & \ \ x>a
%       \end{cases}
%\end{flalign*}
%[width=80mm,scale=0.7]
%\begin{figure}[H]
%  \includegraphics[width=\linewidth]{odd_finite.eps}
%  \caption{$z_0=0.1\pi,0.5\pi, 3\pi,7\pi$}
%  \label{fig4}
%\end{figure}
\end{document}












                                     
                                     



