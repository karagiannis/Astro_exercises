%\documentclass[twocolumn]{article}
\documentclass[./exercises.tex]{subfiles}
\begin{document}

\textit{\textbf{Inlämning 3  } }
\begin{enumerate}

\item \textit{17.41. What are the most prominent absorptions lines you would expect to find in the 
spectrum of a star with a surface temperature of}\\ 
\textit{(a) 35,000 K}\\
\textit{(b) 2800 K }\\
\textit{(c) 5800 K }\\

According to \textit{skyserver.sdss.org}\footnote{\url{http://skyserver.sdss.org/dr1/en/proj/basic/spectraltypes/lines.asp}}
the relationship between star temperature and absorption lines is as follows
\begin{center}
\begin{tabular}{ |c|c|c| } 
 \hline
Spectral type      &Temperature (K)   	& Spectral lines     	\\ 
O                  &28 000-50 000       &Ionized Helium \\
B                  &10 000-28 000       &Helium, some Hydrogen\\
A                  &7500-10 000         &Strong Hydrogen, some ionized metals\\
F                  &6000-7500           &Hydrogen, Kalcium and Iron\\
G                  &5000-6000           &Neutral and ionized metals, especially Kalcium\\
K                  &3500-5000           &Neutral metals, Sodium\\
M                  &2500-3500           &Strong Titanium Oxide, Very strong Sodium\\
 \hline
\end{tabular}
\end{center}

High temperature causes electron-ion collissions (phonons) which excites the electron
to higher energy levels or bands with respect to solids, the electron after a while recombines
or relaxes to lower energy levels.
In the outer plasma gas layer of stars we can possibly
also expect light from vibrational spectra of molecules as molecules shift between different vibrational modes
 as well as light emission from electron exictation-recombination in atoms and ions(except in Hydrogen ions).
Adding the dopplershift due to the motion of individual ions, atoms and molecules because of collissions with each other
we can expect the emission of a continous spectrum of light from stars.
However the outmost coldest gaslayer will give rise to absorption spectra revealing to us its composition
and therefore indirectly the composition of the inner core of the star - making the reasonable
assumption that elements found in the core will also be present in the outmost layer.

(a) At 35 000 degrees Kelvin absorption lines of ionized Helium
would be visible which corresponds to electron excitation
at the wavelength 4400 Ångström according to the mentioned site.\\

(b) Absorption lines visible at stars of temperature 2800 K is
because of the excitation of Sodium and Titanium Oxide\footnote{\url{https://en.wikipedia.org/wiki/Titanium_oxide}}
where it can take quite a few different oxidation numbers and thus a multitude
of absorption lines 4900 - 5200, 5400 - 5700, 6200 - 6300, 6700 - 6900 Ångström\\
Sodium seems to have its most common absorption line at 5800 Ångström according to the mentioned
site.\\

(c) 5800 degrees Kelvin corresponds to Neutral and ionized metals, especially Kalcium
which according to the site mentioned are at 3800 - 4000 Ångström.\\



\textit{Briefly describe why these stars have such different spectra even though they have 
essentially the same chemical composition.  
(Tips: Figure 17-11 samt Table 17-2 i boken är användbara)}
This fact contradicts somewhat what has been reasoned in the answer to the previous question
but can be understood from the following graph\footnote{\url{https://uwm.pressbooks.pub/astronomy/chapter/chapter-17-section-17-3-the-spectra-of-stars-and-brown-dwarfs/}}
\begin{figure}[H]
  \includegraphics[width=\linewidth]{OSC_Astro_17_03_Absorption.jpg}
  \caption{Star temperature determines the absorption lines}
  \label{fig:fig1}
\end{figure}
Hydrogen absorption lines will not be visible in very hot or very cold stars
or might be weak in cold stars.
In hot stars the hydrogen has lost its electron and therefore there can exist
no emission or absorption from/to higher electron energy levels.\\

In the cold outer gas layer of a cold star we expect the electron in the Hydrogen atom 
to reside in its ground state. The radiated electromagnetic energy from the core of a cold star
does no radiate enough power(few photons per second) at wavelengths close to the wavelength
corresponding to excitation of the electron from its ground state to its first excited state, which
is at 122 nm -far UV\footnote{\url{https://en.wikipedia.org/wiki/Hydrogen_spectral_series}}.
A weak absorption line corresponds to few atoms being excited by the passing through EM radiation and
that the radiated power at wavelengths near the excitation wavelength is low.

The graph at figure \ref{fig:fig1} must be understood as the display of the dominant part of the absorption
spectra.\\

At the higest temperature class Hydrogen lines are not visible for the reason just mentioned,
so we only see absorption lines from ionized Helium but not because ionized metals are not present in the gas
but because the ions are so heavily ionized and the elcetrons left bind very hard to the almost unshielded core 
that it probably takes at least soft X-ray radiation to excite inner electrons and therefore
we see no absorption lines from ionized metals in the optical spectra.\\

When temperature has dropped to the next consecutive temperature class,
Helium is no longer ionized or ionized to a lesser degree and we have more neutral Helium
and hence and we get the main contribution of the absorption line from neutral Helium.
Ionized metals are still there but still don't
contribute in the optical absorption spectrum for the same previous mentioned reason - the energy
in the optical spectra photons is not enough to excite a near core electron to a higher shell.\\

At the next consecutive lower temperature class
the absorption spectrum happen to be dominated by Hydrogen but not because Helium is not there but because
individual photon energies of the incident radiation which peaks in wavelength according to the Wien displacement law
do not fit for the excitation of Helium that much.\\

The absorption lines from ionized metals finally can appear in the optical spectrum
because they no longer suffer heavy ionization and thus absorption due to
transitions between outer shells of the ion can appear in the optical spectra.\\

When the temperature insn't high enough to ionize metals or ionized to a lesser degree then
absorption lines of neutral metals who have optical spectrum electron transition photon energies appear as dominating the absorption spectrum.\\

Same explanation with Titan Oxide when the temperature is low enough to allow it to form
and the sort has electron transitions photon energies in the optical spectra then detection of 
optical spectrum absorption lines are being made possible.\\

Same materia present more or less in all stars but temperature determines which elemement will
dominate the absorption spectra.\\

\item\textit{18.8 Why is the daytime sky blue? 
Why are distant mountains purple?
Why is the Sun red at sunset and sunrise?
And how does that relate to the bluish color of reflection nebulae and 
the process of interstellar reddening?}\\


The sky looks blue during the day because the blue wavelenghts of the light which enters at a steep angle are scattered against the air molecules
and eventually reach the eye of the observer and the rest of the EM radiation including
the blueish not scattered penetrate the atmosphere and scatter on objects which enter our
eyes and we see colored objects around us.
A``red'' box absorbes all the optical wavelengths except the ca 700nm wavelengths which is scattered and enters they eye
of the observer
OR it is the case that the ``red'' wavelengts causes electron excitation and recombination emitting
700 nm light.\\
I don't really know which is the case but would really like to know,
so if you teacher know, please inform me.\\

Light which does not enter the earth's atmosphere in a steep angle causes different
optical phenomena due to the longer path travelled in the atmosphere.
The reddish color of the sun at sunset and sunrise and the redness of the sky at dawn
is due to longer path traveled by the light in which only the longer wavelengths are less 
suceptible to back scattering due to dust and air molecules.
Longer wavelengths can't ``see'' smaller particles which is
the same reason the military want to increase radar wavelenghts when wanting to detect vessels which are stealth adapted
to defelct the radar wave in non-detectable directions - away from the radar reciever.\\

\begin{figure}[H]
  \centering
  \includegraphics[scale=0.2]{HMSVisby.jpeg}
  \caption{Geometric stealth structures able to deflect typical size radarwavelengts}
  \label{fig:fig2}
\end{figure}

Small radarwave lengths fit in deflection plates but longer wavelentghs can't ``see''
the angular fitted plate in the structure.
If the whole deflection structure is of the size of the radar wavelenght (optimally $\lambda/2$)
it will appear as a resonant structure to the incident plane wave and be reflect straight back to the radar antenna.
Wavelenghts much longer than the ship as a whole will also not ``see'' the ship at all but continue
along its path.

A similar explanation can be given for distant mountains appearing purple, 
which must be due to the fact that when the mixing of the colors occurs
in the eye, the red part of the light is disproportionate comparing to
the shorter wavelenghts which have been scattered to a larger extent in comparison.\\

Regarding the bluish color of reflection nebulae and 
the process of interstellar reddening care must be taken considering the distance of the path
of the light in the earth's atmosphere with respect to the previous discussion.\\ 
The blueish clor of reflection nebulae can according to the above discussion be explained
as the dust particles being of the size of the blueish wavelengths and thus becoming resonant
structures for the incident light waves.
Regarding the insterstellar reddeing one should
ascribe this to the dopplershift of the light from astronomical objects as they move away from
the observer after ruling out reddening due to longer passage through the earth's atmosphere.\\
\\





\item\textit{18.16 What is an evolutionary track?
 How can evolutionary tracks help us interpret the H-R diagram?}\\

An evolutionary track is the path in the Hertzprung-Russel diagram
taken by a star from its birth to the main sequence and then leaving the main sequence
to its final state\footnote{\url{https://web.njit.edu/~gary/321/Lecture13.html}}.

\begin{figure}[H]
  \centering
  \includegraphics[scale=0.4]{Tracks.png}
  \caption{The path from its dust cloud of birth to the main sequence depends on the mass of the object}
  \label{fig:fig3}
\end{figure}

\item\textit{19.4 Why do high-mass main-sequence stars have shorter lifetimes than those of lower 
mass?}\\

Because they burn off their fuel at a higher rate because of the increased compression
because of the higher gravitation in the core. The higher fuel burning rate
correpsonds to a higher temperature and therfore higher luminosity.
The rate of fuel consumption is supposedly proportional to mass $m^{3.5}$\footnote{\url{http://ganymede.nmsu.edu/cwc/Teaching/ASTR110/PPT2/Lecture11.html}}
 
\item\textit{19.10 Explain why Earth is expected to become inhospitable to life long before the Sun 
becomes a red giant.}\\

Today the sun burns hydrogen in its core and Helium surrounds the core. 
There is a gravitational equilibrium between Helium and inner core Hydrogen which determines the size of the sun.
As the Hydrogen burning off in the core turning into Helium the core, it supposedly lessens
the gravitational pull which in turn inflates the size of the sun.
 As the sun inflates the surface of the sun comes closer Earth
resulting in increasing Earth temperatures.\\
As the surrounding Helium collapses into the core it is
believed that outer shell Hydrogen will ignite and therefore will result in even higher
sun surface temperaturess and thus also higher luminosites
which is believed to be a catastrophy for life on Earth.All this is expected to happen
before the sun engulfs the earth as a red giant.


\item\textit{20.24 Sketch the evolutionary track of the Sun on the H-R diagram from when it leaves the 
main-sequence to when it becomes a white dwarf.  Approximately how much mass will it 
have as a white dwarf and where did the lost mass go}\\

The path for the sun in the H-R diagram from leaving the main sequence, experiencing 
triple-alpha processes, the Helium flash, passing the horizontal branch and becoming
a white dwarf looks something like this\footnote{\url{https://www.e-education.psu.edu/astro801/content/l6_p3.html}}
\begin{figure}[H]
  \centering
  \includegraphics[scale=0.4]{completetrack_KL.jpg}
  \caption{Evolution track of the sun leaving the main sequence and becoming a white dwarf}
  \label{fig:fig4}
\end{figure}

The sun will lose half its mass
shedding its outer layer as a planetary nebula after the red giantphase\footnote{\url{https://www.universetoday.com/25669/the-sun-as-a-white-dwarf-star/}}


\item\textit{22.8 The Galactic halo is dominated by Population II stars, whereas the Milky Way disk
contains primarily population I stars. In which part of the galaxy has star formation taken
place recently? Please explain your answer}

The starformation takes place in spiral arms where clouds of dust and gas particles which are
remenants from supernovae events and lower mass star shedding of material at the end of the life.
These lous gravitated together and thus stars are formed. The assembly of clouds seems to be associated
with associated with the rotation of the spiral galaxy which makes it possible for distinct cloud regions
to meet up and form larger mass regions. The reason that starformation occur in the arms and 
not inbetween the arms seem to be that in between the arms space is occupied by dark matter. 


\item\textit{22.33 In our galaxy, why are stars of spectral class O
and B only found near the spiral
arms? Is the same true for stars of other spectral classes? Explain why or why not.}

Because this is the place where gas and particle clouds have space to assemble, sandwiched it seems between
dark matter fillig up the space between the spiral arms and the stars which together build up
the shape of the arm. The reason class
O and B are found is because they are massive stars which are form and die rlatively quickly.
Lower mass stars correspond to lower luminosities, are older and are therefore the objects which together
give rise to the shape of the spiral arms.

\item\textit{22.45 Another student tells you that the Milky Way is made up “mostly of stars”? Is this
statement accurate? Why or why not?}
It is estiatmated\footnote{\url{https://en.wikipedia.org/wiki/Dark_matter}} that the universe
consists up to 85\% of dark matter so said out of respect of that statement the answer is that the
student is wrong. If we constrain ourself to the visible part of the Milky way we must first
decide if we should compare sum volume of stars compared to volume of gas and dust particles or
total mass of stars to total mass of gas and dust particles. If we compare masses then the student
is right and if we compare volume the student which is unlikely, he might also be right depending
on how we would treat the galactic halo. See picture from
\url{https://phys.org/news/2023-01-astronomers-distant-stars-galaxy-halfway.html}
\begin{figure}[H]
  \centering
  \includegraphics[scale=0.4]{astronomers-find-the-m.jpg}
  \caption{Preicted look of halo of the Milky Way}
  \label{fig:fig4}
\end{figure}

\item\textit{23.10 Which is more likely to have a blue color, a spiral galaxy or an elliptical galaxy?
Please explain why}
 
The blue color is more likely to appear in spiral galaxies because elliptical galaxies
are thought to be formed from colliding spiral galaxies and are therefore more likely to be more
red in color indicating older population of older stars whereas the blue color stems from high mass rapidly
forming short life span high mass stars.

\textit{The typical ages of the stellar populations of elliptical and spiral galaxies provide evidence for this theory,
because the stars in elliptical galaxies are typically
much older and redder than those in spiral galaxies.}\footnote{\url{https://esahubble.org/wordbank/elliptical-galaxy/}}


 

\end{enumerate}


%\underbrace{}

% \hspace{1em}

%\begin{enumerate}[label=(\alph*)]
%\end{enumerate}

%$$
%  A = 
%  \begin{bmatrix}
%    1 & 0  & 2i\\
%    2i & 0 &  -4\\
%    -i &  0 & -2i\\
%  \end{bmatrix}
%$$

%\begin{flalign*}
%  A = 
%  \begin{bmatrix}
%    1 & 0  & 2i\\
%    2i & 0 &  -4\\
%    -i &  0 & -2i\\
%  \end{bmatrix}
%\end{flalign*}


%\begin{flalign*}
%\psi(x) = \begin{cases} Ae^{ikx}+Be^{-ikx} &\ \  x<-a \\
%                        Ce^{\kappa x}+De^{-\kappa x} &\ \ -a < x < a\\
%						Fe^{ikx} & \ \ x>a
%       \end{cases}
%\end{flalign*}
%[width=80mm,scale=0.7]
%\begin{figure}[H]
%  \includegraphics[width=\linewidth]{odd_finite.eps}
%  \caption{$z_0=0.1\pi,0.5\pi, 3\pi,7\pi$}
%  \label{fig4}
%\end{figure}
\end{document}












                                     
                                     



