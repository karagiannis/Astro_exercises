%\documentclass[twocolumn]{article}
\documentclass[./exercises.tex]{subfiles}
\begin{document}

\textit{\textbf{Inlämning 2  } }
\begin{enumerate}
\item \textit{3.4. At approximately what time does the Moon rise when it is
a) a new moon}\\

Enligt internet-sajt\footnote{\url{https://www.almanac.com/when-will-moon-rise-tonight}}
så stiger nymånen nära soluppgången.Nymånen är aldrig synlig förutom vid
solförmörkelse.

\textit{b) a first quarter moon}\\

Första kvarten (halvmåne, höger sida belyst) ca 7 dygn senare är synlig vid skymning

\textit{c) a full moon}\\

Fullmåne ca 7 dagar efter första kvarten vid solnedgången.

\textit{d) a third quarter moon}\\
Ca 7 dygn efter fullmånen är tredje kvarten(halvmåne med vänstra sidan belyst) och
är synlig nära midnatt.



\item \textit{5.5 For each of the following wavelengths, state whether it is in the radio, microwave,
infrared, visible, UV, X-ray or gamma-ray portion of the EM-spectrum}

\textit{a) 2.6 mikrometer}\\
2600nm är långvågigt infrarött - FIR(Far Infra Red).\\
\textit{b) 34 m}\\
Radiofrekvens HF-bandet\\
\textit{c) 0.54 nm}\\
5.4 Ångström är Röntgenstrålning.\\

\textit{d) 0.0032 nm}\\
Gamma-strålning\\
\textit{e) 0.62 mikrometer}\\
620nm är gult ljus\\
\textit{f) 310 nm}
Ultraviolett ljus\\
\textit{g) 0.012 m}
Mikrovågor. X-band\\

\item \textit{5.19 The bright star Bellatrix in the constellation Orion has surface temperature of 21 500 K.
What is its wavelength of maximum emission in nanometers? What colour is this star?}\\

Detta ges av Wiens förskjutningslag\footnote{\url{https://sv.wikipedia.org/wiki/Wiens_lag}}
\begin{flalign*}
\lambda_{max} &=\frac{b}{T}\text{ där}\\
b&=0.00289776 \text{ m}\cdot\text{K}\\
\lambda_{max} &=\frac{0.00289776}{21 500}=1.3478e-07m\\
              &=134 \text{nm - UV-strålning}\\
\end{flalign*}
Denna måste uppfattas som violett-blå för nakna ögat.

\item \textit{5.42 What is the Doppler effect? Why is it important to astronomers?}\\

Antagandet är att ljusets hastighet är konstant för alla våglängder om en ljuskälla lyser mot en
stillastående observatör men att att ljuset beter sig såsom ljudvågor t.ex. från en siren på ett utryckningsfordon
som har högre ton (kortare våglängd) då fordonet närmar sig betraktaren och lägre ton
(längre våglängd) då fordonet avlägsnar sig från betraktaren.
Man kan se en förskjutning av maxima hos vätets emmissionsspektra exemplevis
övergången från $n=3$ till $n=2$. Rydbergs formel ger att den emittterade strålningens
våglängd är 656 nm. Om övergången detekteras såsom exempelvis 658 nm så måste det betyda
att objektet avlägsnar sig.\\

\item \textit{5.56 Use Starry Night to compare the brightness of two similarly sized stars in the
constellation Auriga. The two stars Capella and Delta
Aurigae are indicated in this view. Use the HUD to find the temperature and radius for both of these
stars. Note that these two stars have the same radius but differ in temperature. From these data,
which if these stars is intrinsically brighter and by what proportion (how much brighter)?}\\

Listar de data Stellarium ger
\begin{center}
\begin{tabular}{ |c|c|c| } 
 \hline
\hspace{1em}      & Capella   	& $\delta$ \\ 
 Magnitud         & 0.05  		& 3.7 \\ 
 Absolut Magnitud & -0.54 		& 0.53 \\ 
 Color Index (B-V)& 0.79 		& 1.01 \\ 
 \hline
\end{tabular}
\end{center}
Ju lägre Magnitud desto starkare uppfattas dess intensitet vara mätt från jorden\footnote{\url{https://en.wikipedia.org/wiki/Magnitude_(astronomy)}}.
Användes formeln från Wikipedia som relaterar Magnituder $m$ med intensiteter $I$
uppmätt i $\text{W}/\text{m}^2$ med insatta värden $m_1=0.05$ och $m_{ref}=3.7$ fås
\begin{flalign*}
m_1-m_{ref}&=-2.5\text{log}_{10}\Big(\frac{I_1}{I_{ref}}\Big)\iff\\
10^{\frac{m_{ref}-m_1}{2.5}}&=\frac{I_1}{I_{ref}}\\
\frac{I_1}{I_{ref}}&=10^{\frac{3.7-0.05}{2.5}}=28.4
\end{flalign*}
Vilket betyder att effekttätheten är 28.4 ggr större från Capella relativt $\delta$.\\

Absolut Magnitud är refererat till dess verkliga intensitet och inte hur den uppfattas
p.g.a. av att den befinner sig längre bort än referensstjärnan som jämförelsen görs emot.\\

Capellas radie är $11.98 \pm 0.57$ gånger större än solens radie\footnote{\url{https://en.wikipedia.org/wiki/Capella}}.
och radien hos $\delta$ är ca 11 ggr större än solens\footnote{\url{https://en.wikipedia.org/wiki/Delta_Aurigae}}


Stellarium listar också det s.k. B-V indexet som ett mått på temeperaturen. En varm stjärna
har ett B-V index nära 0 eller negativt värde medan en kall stjärna har ett B-V index närmare 2.\\


\item \textit{5.57. Use Starry Night to examine several different celestial objects and consider the type of
spectrum you would expect each object to radiate.
For each object, decide whether you think it will have a continuous, absorption line, or emission
line spectrum, and explain your reasoning.
(Tips – keep in mind that the light coming from a galaxy is the combined light of hundreds of
billions of stars, and that moonlight is simply re ected sunlight. You may want to consult Figure
5-17 in the book, and read about Kirchhoff’s Laws.)}\\

Förväntan är att ljuset från Andromeda har ett kontiniuerligt spektrum men eftersom Andromeda är så långt bort
så är sannolikheten stor att det finns kalla partikel moln\footnote{Observing Emission/Absorption Lines in Astronomy,\url{https://www.youtube.com/watch?v=lPDY5jiBxJk}} som ljuset passerar på väg till jorden vilket ger
upphov till excitation och absorption där absorption är dominerande eftersom emission kan ske ialla riktningar.
Ljuset från månen är samma bredbandiga ljus som når jorden. Eftersom månen uppfattas som vit så innebär det
att månens yta har ungefär samma reflektionskoefficient för samtliga frekevenser i det optiska spektrat.

M8 galaxen har klassificerats som en emissions galax och en HII-region\footnote{\url{https://en.wikipedia.org/wiki/Lagoon_Nebula}} vilket betyder att det finns
mycket joniserat väte uppenbarligen förväntas ett emissionsspektrum i från vätet.

Att fotografier av M8 blir rosafärgade innebär att effekttätheten är högst omkring 700nm.\\





\item \textit{7.46. Use Starry Night to examine images of the
terrestrial major planets: Mercury, Venus, Earth
and Mars, and also the dwarf planet Ceres.}\\

\textit{(a) Describe each planet’s appearance. From what you observe in each case, is there any way of
knowing whether you are looking at a planet’s surface or at complete cloud cover of the planet?}\\

Avseende om man verkligen ser en planets yta eller ett gasmoln täckandes ytan så borde kriteriet vara
om man ser ojämnheter såsom kratrar eller ej.
 Ett homogent gasmoln som täcker en yta bör ge planeten ett jämnt
bollformat utseende.\\

Merkurius ser gåare ut än månen och från fotgrafiet ser det inte ut att vara några gasmoln på ytan som ser 
ut att vara solid och ha många kratrar.\\

Venus ser ut att ha en gasformig yta eftersom inga meteornedslag är synliga inga ojämnheter
som karaktäriserar en en solid kropp. Enligt Wikipedia så har den en atmosfär av svavelsyra ångor
\footnote{\url{https://en.wikipedia.org/wiki/Venus}}\\

Av den jämna rundheten att döma och de olika färgerna så bör en utomjordisk betraktare
kunna dra slutsatsen att jorden har en atmosfär\\

Mars har kratrar och polarisar. Av polarisarna att döma så borde Mars åttminstone ha en vattenånga
såsom atmosfär. Wikipedia\footnote{\url{https://en.wikipedia.org/wiki/Atmosphere_of_Mars}} beskriver att
Mars har en atmosfär bestående av 95\% Koldioxid, 2.8\% Kväve och 2\% Argon samt spår av vattenånga, syre, kolmonoxid
väte och ädelgaser.\\

Ceres yta verkar ha extremt många kratrar.\\



\textit{(b) Which planet or planets have clouds?}\\

Endast vår jord har moln i enlighet med gängse betydelse av ordet medan hela Venus ser ut att vara täckt
av ett enda moln som täcker hela planetens yta.\\


\textit{(c) Which of these selected planets shows the heaviest cratering?}\\

Anser att Ceres såg mest ``misshandlad'' ut.\\

\textit{(d) Do any of these planets show evidence of liquid water in the planet’s surface today? Do any
show evidence of liquid water on the surface in ancient times}\\

Mars har is vid sina poler och har även spår av floder vilket på engelska benämns ``outflow channels'\footnote{\url{https://en.wikipedia.org/wiki/Outflow_channels}}\\


\textit{(e) What do you notice about Venus’s rotation compared to the rotation of the other planets}\\

Verkar lite oklart vad som skiljer. Egenrotationsperioden är riktigt långsam t.o.m långsammare än dess rotation
kring solen. Listar Siderealt år -SÅ, Sidreal dag -SD.
\begin{center}
\begin{tabular}{ |c|c|c| } 
 \hline
\hspace{1em}       &SÅ(dagar)   	& SD     	\\ 
 Merkurius         & 87.97 			& 1407h30m	\\ 
 Venus             & 224.7 			&5832h28m 	\\ 
 Jorden (B-V)	   & 365.25 		& 23h56m    \\ 
 Mars              &686.97			&24h37m		\\
 Ceres			   &1682.19			&           \\
 \hline
\end{tabular}
\end{center}


\item \textit{7.47. Use Starry Night to examine the gas giant planets Jupiter, Saturn, Uranus, and
Neptune}\\
\textit{(a) Which plant shows the greatest color contrasts in its cloudtops?}\\

Jupiters gasmoln har hög kontrast som måste bero på olika gassammansättningar. Dessa borde väl ha blandats
ut mer efter så lång tid till en mer homogen blandning med avseende på Termodynamikens andra huvudsats kan man väl tycka.
Varför är Jupiter så kontrastrik?\\

\textit{(b) Which planet has the least color contrast in its cloudtops?}\\

Uranus är mest homogen avseende färgen.\\
\textit{(c) What can you say about the thickness of Saturn’s rings compared to their diameter?}\\

De störstaringarna benämns innifrån och ut D, C, B och A. Mittenringarna C och B har störst bredd\footnote{\url{https://en.wikipedia.org/wiki/Rings_of_Saturn\#Major_subdivisions}}\\

\textit{(d) Which planet’s axis is most inclined (lutad) to its orbital plane?}\\

Saturnus lutar mest - $2.485^\circ$ mot ekliptikan.\\

\item \textit{16.65 Use Starry Night to examine simulations of various features that appear on the
surface of the Sun}.

\textit{(a) Which layer of the Sun’s atmosphere is shown in this part of the simulation?}
Gissar på chromosfären eftersom jag inte har tillgång till Starry Nights under Linux och Chrmosfären skall vara
synlig.

\textit{(b) List the different features that are visible in this view of the Sun’s surface, zooming in if
necessary to examine the details of this surface.}\\
Koronan är synlig i det synliga spektrat och chromosfären såsom ett absorptionsspektrum men inte fotosfären vars synliga ljus absorberas på vägen ut.\\


\textit{(c) Select Favourites > Explorations > Chromosphere to see and view from the “surface” of the
Sun. Use the view controls to look around this view. This simulated view of the chromosphere is at
the color of the wavelength of hydrogen light. The opacity of the gas at this wavelength means that
you can see thee structure of the hos chromosphere that lies above the visible surface. Provide a
detailed description of the various features visible in this simulation of the Sun’s surface.}


Kan ej utföra denna utan Starry Nights.\\



\item \textit{16.66 Use Starry Night to measure the Sun’s rotation to display the Sun as seen from about 0.008 au above its surface, well inside the orbit of
Mercury. Use the time controls to stop the Sun’s rotation at a time when a line of longitude on the
Sun makes a straight line between the solar poles, preferably a line crossing a recognisable solar
feature. Note the date and time and then use the time- ow controls to adjust the date and time to
place this selected meridian in this position again.}\\

Solen har en sidreal egenrotation av 25.05 dagar vid dess ekvator
25.38 dagar latitud $16^\circ$ och 34.4 dagar vid polerna.\\

\textit{(a) What is the rotation rate of the Sun as shown in Starry Night? This simulation does not
show on important feature of the Sun, namely its differential rotation, where the equator of this uid
body rotates faster than the polar regions.}\\

Kan ej utföra denna.\\

\textit{(b) To which region of the real Sun does your measured rotation rate refer? In the real Sun, its
differential rotation is thought to generate the magnetic elds and active regions that make the Sun
an active star. Occasional emission of high-energy particles from these active regions can disturb
the Earth’s environment and disrupt electrical transmission systems (sk. solstormar)}\\

Gissar att Starry Nights har implementerat solens rotationstid vid dess ekvator dvs 25 dagar.\\

\textit{(c) Compare this simulated image of the Sun with real images in your textbook or on the
internet. In particular, how does the distribution of sunspots and active regions in this
simulation compare to the distribution of these regions in the real Sun?}

Skriver lite om solfläckar som alternativ till frågan eftersom jag inte kan använda Starry Nights.

Mängden solfläckar följer approximativt en 11 års cykel\footnote{\url{https://en.wikipedia.org/wiki/Sunspot}}\\
Dessa är regioner av förhöjt magnetiskt flöde vilket dämpar värmekonvektion och får dessa fläckar 
därför att erhålla en lägre temperatur är omgivande regioner.
Solfläcken paras ofta med en närliggande region av motsatt magnetisk polaritet.
Indviduella solfläckar och grupper av solfläckar kan bestå allt ifrån några dagar till några månader
men klingar alltid av och försvinner.\\

Solfläckarna är alltid en indikation på intensiv magnetisk aktivitet och är orsaken till laddade partiklar skjuts
ut i s.k. ``solstormar''  som förorsakar elektromagnetiska pulser här på jorden som kan vara så kraftiga att dessa slår ut
elnätet. Den hittills starkaste solstormen vi känner till inträffade 1859\footnote{\url{https://en.wikipedia.org/wiki/Carrington_Event}}\\
1989 slog en solstorm ut Quebecs elnät\footnote{\url{https://en.wikipedia.org/wiki/March_1989_geomagnetic_storm}}

Pulsen verkar förorsakas när källan till magnetfältet plötsligt stoppas\footnote{\url{https://www.youtube.com/watch?v=7nkC8SXzHls}}
i så fall är det precis samma fenomen som när man hastigt bryter strömmen till en spole, magnetfältet
vill inte kollapsa momentant så magnetfältet höjer spänningen och sådant att luften joniseras och en strömförande
ljusbåge bildas ofta går säkringen om den är underdimensionerad.
Om samma slags ledningsväg bryts från solens inre till dess yta så kommer laddade partiklar att
skjutas ut för att förhindra magnetfältets kollaps.
 En solstorm är i så fall ingenting annat än en ``induktiv kick''.




 

\end{enumerate}


%\underbrace{}

% \hspace{1em}

%\begin{enumerate}[label=(\alph*)]
%\end{enumerate}

%$$
%  A = 
%  \begin{bmatrix}
%    1 & 0  & 2i\\
%    2i & 0 &  -4\\
%    -i &  0 & -2i\\
%  \end{bmatrix}
%$$

%\begin{flalign*}
%  A = 
%  \begin{bmatrix}
%    1 & 0  & 2i\\
%    2i & 0 &  -4\\
%    -i &  0 & -2i\\
%  \end{bmatrix}
%\end{flalign*}


%\begin{flalign*}
%\psi(x) = \begin{cases} Ae^{ikx}+Be^{-ikx} &\ \  x<-a \\
%                        Ce^{\kappa x}+De^{-\kappa x} &\ \ -a < x < a\\
%						Fe^{ikx} & \ \ x>a
%       \end{cases}
%\end{flalign*}
%[width=80mm,scale=0.7]
%\begin{figure}[H]
%  \includegraphics[width=\linewidth]{odd_finite.eps}
%  \caption{$z_0=0.1\pi,0.5\pi, 3\pi,7\pi$}
%  \label{fig4}
%\end{figure}
\end{document}












                                     
                                     



